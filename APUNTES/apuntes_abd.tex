\documentclass{article}

\usepackage{amsmath, amsthm, amssymb, amsfonts}
\usepackage{thmtools}
\usepackage{graphicx}
\usepackage{setspace}
\usepackage{geometry}
\usepackage{float}
\usepackage{hyperref}
\usepackage[utf8]{inputenc}
\usepackage[english]{babel}
\usepackage{framed}
\usepackage[dvipsnames]{xcolor}
\usepackage{tcolorbox}

\colorlet{LightGray}{White!90!Periwinkle}
\colorlet{LightOrange}{Orange!15}
\colorlet{LightGreen}{Green!15}

\newcommand{\HRule}[1]{\rule{\linewidth}{#1}}

\declaretheoremstyle[name=Definicion,]{thmsty}
\declaretheorem[style=thmsty,numberwithin=section]{theorem}
\tcolorboxenvironment{theorem}{colback=LightGray}

\declaretheoremstyle[name=Proposition,]{prosty}
\declaretheorem[style=prosty,numberlike=theorem]{proposition}
\tcolorboxenvironment{proposition}{colback=LightOrange}

\declaretheoremstyle[name=Principle,]{prcpsty}
\declaretheorem[style=prcpsty,numberlike=theorem]{principle}
\tcolorboxenvironment{principle}{colback=LightGreen}

\setstretch{1.2}
\geometry{
    textheight=9in,
    textwidth=5.5in,
    top=1in,
    headheight=12pt,
    headsep=25pt,
    footskip=30pt
}

% ------------------------------------------------------------------------------

\begin{document}

% ------------------------------------------------------------------------------
% Cover Page and ToC
% ------------------------------------------------------------------------------

\title{ \normalsize \textsc{}
		\\ [2.0cm]
		\HRule{1.5pt} \\
		\LARGE \textbf{\uppercase{Apuntes ABD}
		\HRule{2.0pt} \\ [0.6cm] \LARGE{} \vspace*{10\baselineskip}}
		}
\date{}
\author{\textbf{Author} \\ 
		B.L.B}

\maketitle
\newpage

\tableofcontents
\newpage

% ------------------------------------------------------------------------------

\section{Control de Acceso}

\begin{theorem}
    \textbf{Administrador:} persona responsable de garantizar la seguridad e integridad de los datos. 
\end{theorem}

\begin{itemize}
	\item Seguridad: previene la revelación, alteración y destrucción de los datos. Los usuarios solo pueden realizar operaciones autorizadas.
	\item Integridad: asegura la exactitud y la validez de los datos (consistentes)
\end{itemize}

Es el responsable principal de la \textbf{seguridad} del SGBD. 
Tiene una cuenta de súper-usuario (forma de acceso protegida) y especifica qué usuarios tienen acceso a qué datos y qué operaciones pueden realizar sobre esos datos.

El acceso se controla mediante privilegios (permisos para realizar operaciones).  

%\begin{proposition}
%    This is a proposition.
%\end{proposition}

%\begin{principle}
 %   This is a principle.
%\end{principle}

% Maybe I need to add one more part: Examples.
% Set style and colour later.

\subsection{Gestión de privilegios}

En esta sección se explorará la forma de crear usuarios y concederles privilegios.
\subsubsection{Crear cuenta de usuario}
 \begin{verbatim}
 	CREATE USER <usuario> IDENTIFIED BY 'contraseña';
 \end{verbatim}
\url{https://dev.mysql.com/doc/refman/8.0/en/create-user.html}
\subsubsection{Borrar cuenta de usuario}
\begin{verbatim}
	DROP USER <usuario>;
\end{verbatim}
\subsubsection{Conceder privilegios}
\begin{verbatim}
	GRANT <privilegio> ON <objeto> TO <sujeto> [WITH GRANT OPTION]
\end{verbatim}
La cláusula "With Grang Option" permite al usuario receptor de un permiso propagarlo a otros usuarios.
\subsubsection{Retirar privilegios}
\begin{verbatim}
	REVOKE [GRANT OPTION FOR] <privilegio> ON <objeto> FROM <sujeto>
\end{verbatim}
 
\subsubsection{Tipos de objeto}
\begin{verbatim}
	 *.* (Global)
	 <BD>.* 
	 <BD>.<tabla>
	 <BD>.<tabla>(columna)

\end{verbatim}
Mostrar permisos otorgados para nuestro usuario:
\begin{verbatim}
	SHOW GRANTS;
\end{verbatim}
Mostrar permisos para otro usuario:
\begin{verbatim}
	SHOW GRANTS FOR <usuario>;
\end{verbatim}

\subsubsection{Ejercicio 1}
Crear BD llamada ``ej1DB"
\begin{verbatim}
	CREATE DATABASE ej1DB;
\end{verbatim}
Crear usuario llamado ``ej1user" y darle permisos para crear tablas en ej1DB y leer, insertar y borrar filas en las tablas de ej1DB
\begin{verbatim}
	CREATE USER ej1user IDENTIFIED BY '1234';
	GRANT CREATE, SELECT, INSERT, DELETE ON ej1DB.* TO ej1user;
\end{verbatim}
Verificar los permisos asignados
\begin{verbatim}
	SHOW GRANTS;
\end{verbatim}

\subsection{Roles}

Un rol es una colección de privilegios con un nombre.
\subsubsection{Crear rol}
\begin{verbatim}
	CREATE ROLE <rol>;
\end{verbatim}
\subsubsection{Asignar privilegios a rol}
\begin{verbatim}
	GRANT <privilegio> ON <objeto> TO <rol>;
\end{verbatim}
\subsubsection{Asignar rol a usuario}
\begin{verbatim}
	GRANT <rol> TO <usuario>;
\end{verbatim}
\subsubsection{Activar y mostrar rol}
\begin{verbatim}
	SET ROLE <rol>;
	SELECT current_role();
\end{verbatim}
\subsubsection{Definir rol por defecto}
\begin{verbatim}
	SET DEFAULT ROLE <rol> TO <usuario>;
\end{verbatim}

\subsubsection{Mostrar privilegios de rol}
\begin{verbatim}
	SHOW GRANTS FOR <usuario> USING <rol>;
\end{verbatim}
\subsubsection{Retirar privilegios}
\begin{verbatim}
	REVOKE <privilegio> ON <objeto> FROM <rol>;
\end{verbatim}
\subsubsection{Eliminar rol}
\begin{verbatim}
	DROP <rol>;
\end{verbatim}
\subsubsection{Ejercicio 2}
Crear una BD "ej2DB" y usuario "ej2user"
\begin{verbatim}
	CREATE DATABASE ej2DB;
	CREATE USER ej2user IDENTIFIED BY '1234';
\end{verbatim}
Crear un rol "ej2rol" con permisos para crear tablas e insertar y leer filas de las tablas.
\begin{verbatim}
	CREATE ROLE ej2rol;
	GRANT CREATE, INSERT, SELECT ON ej2DB.* TO ej2rol;
\end{verbatim}
Asignar el rol al usuario
\begin{verbatim}
	GRANT ej2rol TO ej2user;
\end{verbatim}
%\begin{figure}[htbp]
 %   \center
  %  \includegraphics[scale=0.06]{img/photo.jpg}
  %  \caption{Sydney, NSW}
%\end{figure}

%\subsection{Citation}

%This is a citation\cite{Eg}.

\newpage

% ------------------------------------------------------------------------------
% Reference and Cited Works
% ------------------------------------------------------------------------------

\bibliographystyle{IEEEtran}
\bibliography{References.bib}

% ------------------------------------------------------------------------------

\end{document}